\documentclass[dvipdfmx]{article}
\usepackage{amssymb,amsmath,latexsym}
\usepackage{figtools}
\usepackage{listings}
\lstset{basicstyle={\ttfamily\small},basewidth=0.45em}

\begin{document}
\title{An Example of \LaTeX \; for FigTools and TikZ}
\author{Mitsuhiro Hattori}
\date{\today}
\maketitle

\section{Description}
In order to use \textbackslash figtools, you need to put figtools.tex in the folder and write \textbackslash usepackage\{figtools\}.
You can specify size as an option of \textbackslash figtools.
Figure~\ref{fig:exampleFigure} is a figure generated by MATLAB FigTools in plotExampleFigure.m with option '-tex'.
MATLAB FigTools requires matlab2tikz tool to generate .tex file. \par
The code to generate Figure~\ref{fig:exampleFigure} is as follows:
\begin{lstlisting}
    \begin{figure}[h]
        \centering
        \figtools[width=80mm, height=40mm]{plot/tex/exampleFigure.tex}
        \caption{The example figure.}\label{fig:exampleFigure}
    \end{figure}
\end{lstlisting}
If you do not specify the value of width and height, the default value for them is 60mm.
The font size will be same as text and caption, no matter what figure size you choose.
\begin{figure}[h]
    \centering
    \figtools[width=80mm, height=40mm]{plot/tex/exampleFigure.tex}
    \caption{The example figure.}\label{fig:exampleFigure}
\end{figure}

\end{document}